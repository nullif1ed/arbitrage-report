\documentclass[12pt]{article}

% ---------------------------------------------------------
% PACKAGES
% ---------------------------------------------------------
\usepackage{amsmath, amssymb, siunitx, graphicx, float, booktabs}
\usepackage[margin=1in]{geometry}
\usepackage{hyperref}
\hypersetup{colorlinks=true, linkcolor=blue, urlcolor=blue, citecolor=blue}
\setlength{\parskip}{0.8em}
\setlength{\parindent}{0pt}

% ---------------------------------------------------------
% TITLE
% ---------------------------------------------------------
\title{\textbf{High-Frequency Arbitrage Analysis between Binance and Coinbase}}
\author{Nicholas Nguyen, Jay Patel, Nelson Siu}
\date{\today}

% ---------------------------------------------------------
% DOCUMENT
% ---------------------------------------------------------
\begin{document}
\maketitle
\tableofcontents
\newpage

% ---------------------------------------------------------
\section{Introduction}
This report presents a complete arbitrage analysis framework designed to detect and exploit short-term mispricings between the Binance and Coinbase cryptocurrency exchanges. While the data used in this version are synthetic and meant for demonstration, the framework is compatible with real-time or historical tick-level data (second- or millisecond-level resolution) when integrated with live APIs.

The objective is to model the spread between prices, identify mean-reversion behavior, and simulate trades based on a rolling z-score strategy. The analysis outputs include spread evolution, signal generation, trade markers, cumulative and mark-to-market profit and loss (P\&L), and return correlation between the two markets.

% ---------------------------------------------------------
\section{Data Description and Preprocessing}
For this demonstration, the data were generated to mimic real BTC/USD pricing behavior across two exchanges with micro deviations to emulate cross-exchange inefficiencies. Each row represents one-minute aligned timestamps in ISO 8601 UTC format.

The preprocessing steps were:
\begin{enumerate}
    \item Align Binance and Coinbase price data by timestamp.
    \item Compute the instantaneous spread: 
    \[
    s(t) = P_{Binance}(t) - P_{Coinbase}(t)
    \]
    \item Calculate rolling mean and standard deviation over a 30-period window to normalize the spread:
    \[
    z(t) = \frac{s(t) - \mu_s(t)}{\sigma_s(t)}
    \]
    \item Generate trading signals when $|z(t)| > 2$ (entry) and close when $|z(t)| < 0.5$.
\end{enumerate}

\begin{table}[H]
\centering
\begin{tabular}{lcc}
\toprule
Column Name & Description & Units \\
\midrule
\texttt{time} & Timestamp (UTC) & ISO 8601 \\
\texttt{binance} & Binance mid-price & USD \\
\texttt{coinbase} & Coinbase mid-price & USD \\
\texttt{spread} & Price difference (Binance - Coinbase) & USD \\
\texttt{spread\_z} & Rolling z-score of spread & unitless \\
\texttt{position} & Current open trade (+1/-1/0) & unitless \\
\texttt{m2m\_pnl} & Mark-to-market P\&L & USD \\
\texttt{cumulative\_pnl} & Realized P\&L & USD \\
\bottomrule
\end{tabular}
\caption{Summary of data columns used in the arbitrage analysis.}
\end{table}

% ---------------------------------------------------------
\section{Exploratory Analysis and Visualization}

\subsection{Price Series}
\begin{figure}[H]
    \centering
    \includegraphics[width=0.9\textwidth]{figure1_price_series.png}
    \caption{Hypothetical Binance and Coinbase BTC/USD price series over time. The minor fluctuations between the two exchanges mimic latency-driven price discrepancies.}
\end{figure}

\subsection{Spread Evolution}
\begin{figure}[H]
    \centering
    \includegraphics[width=0.9\textwidth]{figure2_spread.png}
    \caption{Spread ($P_{Binance} - P_{Coinbase}$) showing persistent oscillation around zero. These deviations represent potential arbitrage entry points.}
\end{figure}

\subsection{Normalized Spread (Z-Score)}
\begin{figure}[H]
    \centering
    \includegraphics[width=0.9\textwidth]{figure3_spread_zscore.png}
    \caption{Z-score of the spread computed over a rolling 30-minute window. High-magnitude z-scores indicate possible arbitrage opportunities when spreads deviate from equilibrium.}
\end{figure}

\subsection{Entry and Exit Markers}
\begin{figure}[H]
    \centering
    \includegraphics[width=0.9\textwidth]{figure4_spread_markers.png}
    \caption{Spread evolution with entry (upward triangles) and exit (downward triangles) markers based on the z-score thresholds. Positive z-score entries indicate selling Binance and buying Coinbase, and vice versa.}
\end{figure}

\subsection{Cumulative Realized P\&L}
\begin{figure}[H]
    \centering
    \includegraphics[width=0.9\textwidth]{figure5_cum_pnl.png}
    \caption{Cumulative realized P\&L from executed trades. Despite transaction cost neglect, the shape indicates the potential profitability of the mean-reversion arbitrage approach.}
\end{figure}

\subsection{Mark-to-Market (Unrealized) P\&L}
\begin{figure}[H]
    \centering
    \includegraphics[width=0.9\textwidth]{figure6_m2m.png}
    \caption{Mark-to-market (unrealized) P\&L showing fluctuations during open positions. It captures intratrade volatility before final realization.}
\end{figure}

\subsection{Distribution of Trade Profits}
\begin{figure}[H]
    \centering
    \includegraphics[width=0.8\textwidth]{figure7_hist_pnl.png}
    \caption{Histogram of per-trade realized P\&L. The distribution suggests that most trades cluster near zero profit, consistent with small, frequent arbitrage captures.}
\end{figure}

\subsection{Rolling Return Correlation}
\begin{figure}[H]
    \centering
    \includegraphics[width=0.9\textwidth]{figure8_rolling_corr.png}
    \caption{Rolling correlation (30-minute window) of log returns between Binance and Coinbase. Strong co-movement ($r \approx 1$) reflects market efficiency, while short-term dips suggest temporary desynchronization exploitable for arbitrage.}
\end{figure}

% ---------------------------------------------------------
% ---------------------------------------------------------
\section{Critical Delay Estimation}
\label{sec:criticaldelay}

\subsection{Model Framework}
To evaluate the temporal stability of the arbitrage feedback system, we model the spread dynamics between exchanges using a first-order delayed feedback differential equation:
\begin{equation}
    \frac{ds(t)}{dt} = -k\,s(t - \tau) + \eta(t),
\end{equation}
where $k > 0$ represents the mean-reversion or feedback gain, $\tau$ is the latency (reaction delay), and $\eta(t)$ is noise representing stochastic market disturbances.

The equilibrium stability boundary for this delayed differential equation occurs when the characteristic equation
\begin{equation}
    \lambda + k e^{-\lambda \tau} = 0
\end{equation}
admits purely imaginary roots. Setting $\lambda = i\omega$ and separating real and imaginary parts yields
\begin{align}
    k \cos(\omega \tau) &= 0, \\
    \omega - k \sin(\omega \tau) &= 0.
\end{align}
Solving for the smallest oscillatory mode ($\omega = k$, $\sin(\omega\tau) = 1$) gives the \textbf{critical delay}:
\begin{equation}
    \tau_c = \frac{\pi}{2k}.
\end{equation}
For $\tau < \tau_c$, the feedback loop is stable (mean-reverting); for $\tau > \tau_c$, delayed reaction introduces oscillatory instability.

\subsection{Empirical Estimation of Feedback Gain}
We estimate $k$ empirically from the observed spread series using the continuous-time approximation:
\begin{equation}
    \frac{ds}{dt} \approx -k\, s(t),
\end{equation}
which leads to a least-squares estimate
\begin{equation}
    \hat{k} = -\frac{\sum_t s(t)\, \dot{s}(t)}{\sum_t s(t)^2}.
\end{equation}

Using the synthetic aligned dataset generated earlier, we obtain
\[
    \hat{k} \approx -1.00\times10^{-4}\ \text{s}^{-1},
\]
which is negative in this toy example, implying that the model’s simplified feedback assumption does not hold for these random prices. Consequently, $\tau_c$ is undefined (non-physical) here. With real arbitrage data, positive $k$ values are expected, yielding a finite $\tau_c$ that characterizes how much latency the system can tolerate before instability.

\subsection{Grid Search for Delay-Dependent Gain}
To obtain $\hat{k}(\tau)$ for various assumed delays $\tau_0$, we regress
\[
    \frac{s_{t+\Delta}-s_t}{\Delta} \approx -k\, s_{t-\ell},
\]
where $\ell$ is the discrete lag corresponding to $\tau_0 = \ell \Delta$. We compute $\hat{k}(\tau_0)$ across a grid of candidate delays and evaluate the implied $\tau_c(\tau_0) = \pi / (2\hat{k}(\tau_0))$.

\begin{figure}[H]
    \centering
    \includegraphics[width=0.85\textwidth]{figure9_k_vs_tau.png}
    \caption{Estimated feedback gain $\hat{k}(\tau)$ as a function of assumed delay $\tau$. Stable regions correspond to $\hat{k} > 0$.}
\end{figure}

\begin{figure}[H]
    \centering
    \includegraphics[width=0.85\textwidth]{figure10_tau_c_vs_tau.png}
    \caption{Estimated critical delay $\tau_c(\tau)$ computed from $\hat{k}(\tau)$ using $\tau_c = \pi / (2\hat{k})$. Finite $\tau_c$ values indicate the predicted onset of instability due to latency.}
\end{figure}

\subsection{Interpretation}
Figure~\ref{sec:criticaldelay} illustrates how reaction latency can destabilize arbitrage execution. If real exchange data exhibit positive mean-reversion strength ($k>0$), then:
\[
    \tau_c = \frac{\pi}{2k}
\]
quantifies the maximum permissible delay before oscillatory mispricing amplification occurs.

Empirically, a high $\hat{k}$ implies stronger correction of spreads and smaller $\tau_c$ (system more sensitive to delay), while small $\hat{k}$ implies slower feedback and larger allowable latency.

This theoretical framework connects directly to algorithmic trade execution speed, co-location strategies, and cross-exchange latency optimization.


\section{Preliminary Results and Insights}
Even in the synthetic test case, the strategy identifies clear high-z-score moments corresponding to profitable opportunities. The framework demonstrates that:
\begin{itemize}
    \item The spread oscillates around zero, confirming short-term mispricings.
    \item Z-score normalization provides an effective basis for entry/exit signal generation.
    \item The resulting P\&L trajectory shows positive drift under ideal conditions (no slippage, latency, or fees).
\end{itemize}

This foundation can be applied to real tick data with adjustments for:
\begin{itemize}
    \item Exchange latency and order-book microstructure.
    \item Maker-taker fee asymmetry.
    \item Transfer constraints between exchanges.
\end{itemize}

% ---------------------------------------------------------
\section{Conclusion and Next Steps}
This project establishes a complete arbitrage modeling workflow:
\begin{enumerate}
    \item Align tick-level data between exchanges.
    \item Compute spreads and normalized z-scores.
    \item Generate and backtest mean-reversion-based trading signals.
    \item Visualize dynamics and profitability metrics.
\end{enumerate}

\textbf{Future work} will incorporate:
\begin{itemize}
    \item True millisecond-resolution tick data from Binance and Coinbase APIs.
    \item Latency modeling and empirical estimation of delay sensitivity.
    \item Dynamic position sizing, risk constraints, and transaction cost modeling.
    \item Integration with a live data streaming system for real-time arbitrage signal generation.
\end{itemize}

\vfill
\noindent\textbf{Files Included:}
\begin{itemize}
    \item \texttt{binance\_forced\_string.csv}
    \item \texttt{coinbase\_forced\_string.csv}
    \item \texttt{combined\_with\_spread\_and\_signals.csv}
    \item Figures 1--8 as PNGs
\end{itemize}

\end{document}
